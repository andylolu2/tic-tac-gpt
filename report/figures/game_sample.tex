\begin{figure}
    \centering
    \renewcommand{\arraystretch}{1.3}
    \begin{tabular}{lccccccc}
        Game
         & \cbox{\tictactoe{
                n,n,n,
                n,n,n,
                n,n,n,
            }}
         & \cbox{\tictactoe{
                _,X,_,
                _,_,_,
                _,_,_,
            }}
         & \cbox{\tictactoe{
                _,X,_,
                _,O,_,
                _,_,_,
            }}
         & \cbox{\tictactoe{
                _,X,_,
                _,O,_,
                _,_,X,
            }}
         & \cbox{\tictactoe{
                _,X,_,
                _,O,O,
                _,_,X,
            }}
         & \cbox{\tictactoe{
                X,X,_,
                _,O,O,
                _,_,X,
            }}
         & \cbox{\tictactoe{
                X,X,_,
                O,O,O,
                _,_,X,
        }}                   \\
        \midrule
        Input
         & \texttt{[B]}
         & \texttt{1}
         & \texttt{4}
         & \texttt{8}
         & \texttt{5}
         & \texttt{0}
         & \texttt{3}        \\
        Target
         & \texttt{1}
         & \texttt{4}
         & \texttt{8}
         & \texttt{5}
         & \texttt{0}
         & \texttt{3}
         & \texttt{[O]}      \\
    \end{tabular}
    \caption{Example of a \ttt game and its corresponding inputs and next-token prediction targets. The numbering of the board positions are shown in the first board (top left). \texttt{[B]} is the special beginning-of-sequence token and \texttt{[O]} denotes the end of the game with the result ``\texttt{O} wins''.}
    \label{fig:game-sample}
\end{figure}